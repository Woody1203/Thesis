\part{Comparison} \label{part:Comparison and implementation}

\chapter{comparisons and implementations}

\section{t-SNE}
There are 4 most important parameters for t-SNE, the basic ideas of each parameters are: 

\begin{enumerate}[1)]
\item $perplexity$: number of nearest neighbors for each point
\item $n\_iter$: maximum number of iterations for optimization
\item $early\_exaggeration$: the tightness inside clusters and the distance inbetween
\item $min\_grad\_norm$: the threshold for stopping the optimization 
\end{enumerate}
\\

\noindent When we are analyzing the result, we should consider that the high-dimensional sphere was modeled by probability graph, the t-SNE algorithm adapts its "distance" concept to changes in regional density in the data set. As a result, it naturally expands dense clusters and shrinks sparse clusters, making the clusters roughly uniform in size. It also makes it difficult to see the relative size of clusters in the t-SNE graph, and the distance between clusters in low dimensions does not represent the distance between clusters in high dimensions.\\

\subsection{$perplexity$}

\noindent The first adjustable parameter of t-SNE represent the number of neighbors each point t-SNE considers. It explains how to maintain a balance between the local and global aspects of the data. The perplexity value has a complex effect on the generated pictures, but the performance of SNE is quite reliable for changes in complexity, and the typical value is between 5 and 50\cite{ref9}. Also, since the perplexity means the number of the neighbors, it should not be higher than the number of data points.\\

\noindent We can try different potential perplexity values in datasets with different instances. At the same time, since the algorithm using stochastic method, the parameter $random\_state$ is set to 1. This parameter determines the random number generator so that it pass an int mark for reproducible results across multiple function calls. we get the figures below:\\
\\
\\
% plt diff cluster result with diff perp for three dataset.

As we can seen from the graph:\\


\noindent Taking into account the experimental results and the inspiration of the search about the optimal number of neighbor from K nearest neighbour algorithm\cite{ref12}, we can simply set perplexity to $n^{0.5}$, where n stand for number of instances.\\


\subsection{$n\_iter$}

$n\_iter$ is the maximum number of iterations for the algorithm. The graph observed above are all generated with 1000 iteration which is the default value of $n\_iter$. In principle, the higher $n\_iter$ gives the better result. However, it may be very time consuming when it comes to big dataset. In contrast, if $n\_iter$ is too small, the algorithm does not have enough iteration to generate the convincing clusters in LD. The most important thing is to reach a stable configuration. The figure shows how the result changes with $n\_iter$ grows.

% 上图显示了在困惑度30下的五个不同的运行。前四个在稳定之前已停止。经过10、20、60和120步后,您可以看到带有簇的一维甚至点状图像的布局。如果您看到t-SNE图上有奇怪的“挤压”形状,则该过程可能停止得太早。不幸的是,没有固定数量的步骤可以产生稳定的结果。不同的数据集可能需要不同数量的迭代才能收敛。

% 之后还有perp与n_iter之间的heatmap,解释一波